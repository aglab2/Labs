\documentclass[a4paper,12pt]{article} %размер бумаги устанавливаем А4, шрифт 12пунктов
\usepackage{ucs}
\usepackage{amsmath}
\usepackage{mathtext}
\usepackage{pstool}
\usepackage{color}
\usepackage{amsfonts}
\usepackage{amssymb}
\usepackage{amsthm} %proof подключаем
\usepackage{mathtools}
\usepackage[utf8x]{inputenc} % Включаем поддержку UTF8
\usepackage[russian]{babel}  % Включаем пакет для поддержки русского языка
\date{}
\author{}
\usepackage{mathrsfs}
\usepackage{graphicx} %хотим вставлять в диплом рисунки?
\usepackage{caption}
\usepackage{sidecap}
\usepackage{wrapfig}
\usepackage{pgf,tikz}
\usetikzlibrary{patterns}
\usetikzlibrary{arrows}
\usepackage{pgf, tikz}
\usepackage[left=3cm,right=2cm,
    top=2cm,bottom=2cm,bindingoffset=0cm]{geometry}
\usetikzlibrary{arrows}

\usepackage{setspace}
\graphicspath{{images/}}%путь к рисункам
\DeclareGraphicsExtensions{.pdf,.png,.jpg}

\makeatletter
\renewcommand{\@biblabel}[1]{#1.} % Заменяем библиографию с квадратных скобок на точку:
\makeatother

\newcommand{\divisible}{\mathop{\raisebox{-2pt}{\vdots}}}

\begin{document}
\begin{flushleft}
{Пособие №14}
\hfill
{\bf ``Алгебраические уравнения, системы и неравенства''}
\end{flushleft}				

В данной статье мы не будет приводить много теории, поскольку эти темы уже
были рассмотрены в отдельных статьях, а просто покажем, некоторые примеры, 
в которых всем этим будем оперировать, и начнем мы с систем уравнений и неравенств.

Пытаться выделить здесь какие-то особые способы решения систем бессмысленно, 
т.к. их решение как и способ напрямую зависит от тех уравнений или неравенств, которые они содержат.
Иногда некоторые системы можно привести к эквиватентной путем каких-то преобразований, можно
максимально их упрощать используя функциональный подход, например метод итераций (см. пособие 13 и 8), либо
тригонометрические тождества и так далее. Еще очень часто бывает удобно использовать метод
оценок и некоторые неравенства. Рассмотрим для начала системы алгебраических уравнений.

\begin{center}
{\large Примеры}
\end{center}

\label{Problem1}
\underline{Задача №1}
Решить систему $\begin{dcases}
	\sqrt{8y-x}+x=2 \\
	\sqrt{3y-x}+x+y=2
\end{dcases}$\newline
Решение: Сделаем замену 
$\begin{dcases}
	u = \sqrt{8y-x} \\
	v = \sqrt{3y-x}
\end{dcases}$,
$\begin{dcases}
	u^2 = 8y-x \\
	v^2 = 3y-x
\end{dcases}\Rightarrow u^2-v^2=5y$.
Вычтем из первого уравнения второе и получим: $\sqrt{8y-x}=\sqrt{3y-x}+y \Leftrightarrow$
$5u=5v+u^2-v^2 \Leftrightarrow (v-u)(5-(v+u))=0 \Rightarrow $
$\left[ \begin{gathered}
	v=u \hfill \\
	v+u = 5
\end{gathered} \right.$
Если $v=u$, то $y=0$, тогда из первого уравнения $\sqrt{-x}+x=2 \Rightarrow \varnothing$.
Если $u+v=5$, то  
$\begin{dcases}
	\sqrt{8y-x}+x=2 \\
	\sqrt{3y-x}+\sqrt{8y-x}=5
\end{dcases} \Leftrightarrow$
$\begin{dcases}
	8y-x=(2-x)^2 \\
	3y-x=(3+x)^2
\end{dcases}, $
$\begin{dcases}
	y=\dfrac{(2-x)^2+x}{8} \\
	y=\dfrac{(3+x)^2+x}{3}
\end{dcases} \Rightarrow$
$3(2-x)^2+3x=8(3+x)^2+8x \Rightarrow x=-14.$
Имеем 
$\begin{dcases}
	x = -14 \\
	y = \dfrac{121}{4}
\end{dcases}$
Ответ: $\left( -14; \dfrac{121}{14} \right)$

\label{Problem2}
\underline{Задача №2}
Решить систему $\begin{dcases}
	2x^2-xy-3y^2=0 \\
	x^2-3xy+2y^2=-1
\end{dcases}$\newline
Решение: Если $x=y=0$, то уравнение первого уравнения пара $(0;0)$, но она не
является решением второго уравнения, следовательно $x\ne0$, и $y\ne0$, тогда
поделим оба уравнения на $y^2 \Rightarrow$
$\begin{dcases}
	2\left( \dfrac{x}{y} \right)^2-\dfrac{x}{y}-3 = 0 \\
	\left( \dfrac{x}{y} \right)^2 - 3\dfrac{x}{y} + 2 = -\dfrac{1}{y^2}
\end{dcases}$ 
Пусть $\dfrac{x}{y}=t$, решим первое уравнение: $2t^2-t-3=0 \Rightarrow t_{1,2}=-1; \dfrac32$.
Тогда получим: 
$\begin{dcases}
	\dfrac{x}{y} = -1 \\
	1+3+2=-\dfrac{1}{y^2}
\end{dcases} \Rightarrow \varnothing$
или 
$\begin{dcases}
	\dfrac{x}{y} = \dfrac32 \\
	\dfrac94-\dfrac92+2=-\dfrac{1}{y^2}
\end{dcases} \Rightarrow $
$\begin{dcases}
	\dfrac1{y^2}=\dfrac14 \\
	\dfrac{x}{y} = \dfrac32
\end{dcases} \Rightarrow$
$\begin{dcases}	
	y=-2 \\
	x=-3
\end{dcases}$ или
$\begin{dcases}
	y=2 \\
	x=3
\end{dcases}$\\
Ответ: $(-3; -2); (3;2)$

\pagebreak

\label{Problem3}
\underline{Задача №3} Решить систему 
$\begin{dcases}
	\sqrt{x+y}+\sqrt{x+2y}=10 \\
	\sqrt{x+y}+2x+y=16
\end{dcases}$\newline
Решение: Пусть $\sqrt{x+y}=u$, $\sqrt{x+2y}=v$, решим уравнение \\
$\begin{dcases}
	x+y=u^2 \\
	x+2y=v^2
\end{dcases} \Rightarrow$
$\begin{dcases}
	y = v^2-u^2 \\
	x = u^2-2-v^2
\end{dcases} \Rightarrow$
$\begin{dcases}
	y = v^2-u^2 \\
	x = 2u^2-v^2
\end{dcases}$, тогда исходная система равносильна
$\begin{dcases}	
	u+v=10 \\
	u+u^2+2u^2-16 = 0
\end{dcases} \Leftrightarrow$
$\begin{dcases}
	3u^2-v^2+u-16 = 0 \\
	v = 10-u
\end{dcases} \Leftrightarrow$
$\begin{dcases}
	2u^2+21u-116=0 \\
	v = 10-u
\end{dcases} \Rightarrow$
$\begin{dcases}
	u = 4 \\
	v = 6
\end{dcases} \Rightarrow$
$\begin{dcases}
	x=-4 \\
	y=20
\end{dcases}$

Часто встречаются симметрические, которые имеют вид:
$\begin{dcases}
	f(x,y) + g(x,y) = 0 \\
	f(x,y) g(x,y) = b
\end{dcases}$, например, 
$\begin{dcases}
	x+y=a \\
	xy=b
\end{dcases}$.
Многочлены $x+y$ и $xy$ в левых частях уравнений системы являются простейшими
симметрическими многочленами, а любой симметрический многочлен от $u$ и $v$, где
$u=x+y$ и $v=xy$. При решении симметрической системы часто приходится 
выражать через $u$ и $v$ многочлены вида $P_n(x,y)=x^n+y^n$. Приведем некоторые примеры:\\
$P_2(x) = x^2+y^2=(x+y)^2-2xy \equiv u^2-2v$\\
$P_3(x) = x^3+y^3=(x+y)^3-3x^2y-3xy^2=(x+y)^3-3xy(x+y)\equiv u^3-3uv$ \\
$P_4(x) = x^4+y^4=(x+y)^4-4x^3y-6x^2y^2-4xy^3=(x+y)^4-4xy(x^2+y^2)-6x^2y^2=(x+y)^4-4xy(x+y)^2+2x^2y^2 \equiv u^4-4vu^2+2v^2$\\
$P_5(x) = x^5+y^5\equiv u^5-5u^3v+5uv^2$\\
В общем случае имеет место формула:
\end{document}
