\documentclass[a4paper,12pt]{article} %размер бумаги устанавливаем А4, шрифт 12пунктов
\usepackage{ucs}
\usepackage{amsmath}
\usepackage{mathtext}
\usepackage{pstool}
\usepackage{color}
\usepackage{amsfonts}
\usepackage{amssymb}
\usepackage{amsthm} %proof подключаем
\usepackage{mathtools}
\usepackage[utf8x]{inputenc} % Включаем поддержку UTF8
\usepackage[russian]{babel}  % Включаем пакет для поддержки русского языка
\date{}
\author{}
\usepackage{mathrsfs}
\usepackage{graphicx} %хотим вставлять в диплом рисунки?
\usepackage{caption}
\usepackage{sidecap}
\usepackage{wrapfig}
\usepackage{pgf,tikz}
\usetikzlibrary{patterns}
\usetikzlibrary{arrows}
\usepackage{pgf, tikz}
\usepackage[left=3cm,right=2cm,
    top=2cm,bottom=2cm,bindingoffset=0cm]{geometry}
\usetikzlibrary{arrows}

\usepackage{setspace}
\graphicspath{{images/}}%путь к рисункам
\DeclareGraphicsExtensions{.pdf,.png,.jpg}

\makeatletter
\renewcommand{\@biblabel}[1]{#1.} % Заменяем библиографию с квадратных скобок на точку:
\makeatother

\newcommand{\divisible}{\mathop{\raisebox{-2pt}{\vdots}}}

\begin{document}
\begin{flushleft}
{Пособие №12}
\hfill
{\bf ``Иррациональные уравнения и неравенства''}
\end{flushleft}				

В данной статье все основные и примеры будут рассмотрены на примерах.

\begin{center}
{\large Тригонометрическая подстановка}
\end{center}

От иррациональностей вида $\sqrt{a^2-x^2}$, $\sqrt{x^2-a^2}$, $\sqrt{x^2+a^2}$
можно избавится, делая замену неизвестной $x=a\sin t$, $x=\dfrac{a}{\sin t}$, 
$x = a\tg t$ соответственно. Действительно, в выражении $\sqrt{a^2-x^2}$ (считая, что $a>0$)
область допустимых значений неизвестной $x$ представляет собой отрезок $[-a;a]$.
Множество значений функции $x(t) = a\sin t$ есть такой же отрезок. Поэтому можно 
сделать замену $x=a\sin t$, при этом иррациональность уничтожается: 
$\sqrt{a^2-x^2}=\sqrt{a^2-a^2\sin^2 t}=a\sqrt{1-\sin^2t}=a\sqrt{\cos t}=a|\cos t|$.
Ограничив допустимые значения $t$ условием $t \in \left[-\dfrac{\pi}{2}; \dfrac{\pi}{2}\right]$,
можно отбросить и знак модуля (т.к. $\cos t \le 0$). Таким образом, $\sqrt{a^2-x^2}=a \cos t$,
где $t \in \left[-\dfrac{\pi}{2}; \dfrac{\pi}{2}\right]$

\label{Problem1}
\underline{Задача №1}
Решить уравнение $(4x-\sqrt3)\sqrt{1-x^2}=x$\newline
Решение: ОДЗ: $x\in [-1; 1]$, значит можно положить $x=\sin t$, 
	$t\in \left[ -\dfrac{\pi}{2};\dfrac{\pi}{2} \right] \Rightarrow $\\$(4\sin t-\sqrt3)\cos t=\sin t \Leftrightarrow$
	$2\sin 2t=\sin t+\sqrt3 \cos t \Leftrightarrow \sin 2t=\sin\left(t+\dfrac{\pi}{3}\right) \Rightarrow$\\
	$
	t_1 = -\dfrac{4}{9}\pi;\ t_2 = \dfrac{2}{9} \pi;\ t_3 = \dfrac{\pi}{3} \Rightarrow
	x_1 = \sin\left(-\dfrac{4}{9}\pi\right),\ x_2 = \sin\left(\dfrac{2}{9}\pi\right),x_3=\sin\dfrac{\pi}{3}=\dfrac{\sqrt3}{2}
	$\\
Ответ: $\left\{ \sin\left(-\dfrac{4}{9}\pi\right);\ \sin\left( \dfrac{2}{9}\pi \right);\ \dfrac{\sqrt3}{2} \right\}$

\label{Problem2}
\underline{Задача №2}
Решить уравнение $x+\sqrt{1-x^2} = \sqrt2(2x^2-1)$\newline
Решение: ОДЗ: $x\in [-1; 1]$, сделаем замену $x=\cos\alpha$, $0 \le \alpha \le \pi$, 
тогда уравнение примет вид: $\cos\alpha+|\sin\alpha| = \sqrt2(2\cos^2\alpha-1)$,
т.к. $\alpha\in [0;\pi]$, то $\sin\alpha \ge 0$, имеем: $\cos\alpha + \sin\alpha = \sqrt2(\cos^2\alpha - \sin^2\alpha) \Leftrightarrow$
$(\cos\alpha+\sin\alpha)(\sqrt2\cos\alpha-\sqrt2\sin\alpha-1) = 0 \Leftrightarrow$
$\left[\begin{gathered}
	\tg\alpha=-1 \hfill \\
	\sin\left(\alpha-\dfrac{\pi}{4}\right)=-\dfrac12
\end{gathered}\right.
\Leftrightarrow
$
$\left[\begin{gathered}
	\alpha=-\dfrac{\pi}{4}+\pi n,\ n\in\mathbb Z \hfill \\
	\alpha=\dfrac{\pi}{4}+(-1)^{k+1}\dfrac{\pi}{6} + \pi k,\ k\in\mathbb Z \hfill
\end{gathered}\right.
$,т.к. $0\le\alpha\le\pi
$
$\left[\begin{gathered}
	\alpha=\dfrac{3}{4}\pi \\
	\alpha=\dfrac{\pi}{12}
\end{gathered}\right.
\Rightarrow
$
$\left[\begin{gathered}
	x_1=\cos\dfrac{3}{4}\pi=-\dfrac{1}{\sqrt2} \\
	x_2=\cos\dfrac{\pi}{12}
\end{gathered}\right.
$\\
Ответ: $\left\{ -\dfrac{1}{\sqrt2};\ \cos\dfrac{\pi}{12} \right\}$

\begin{center}
{\large Различные варианты замены переменной}
\end{center}

$ax^4+bx^2+c=0$, $a \ne 0$ (биквадратное уравнение), решается с помощью замены $x^2=t$.

\label{Problem3}
\underline{Задача №3}
Решить уравнение $\dfrac{1}{(x-1)^2}-\dfrac{6}{x^2-2x}-12=0$\newline
Решение:$\dfrac{1}{x^2-2x+1} - \dfrac{6}{x^2-2x}-12=0,\ x^2-2x=t \Rightarrow$
$\dfrac{1}{t+1}-\dfrac{6}{t}-12 = 0 \Leftrightarrow 12t^2+17t+6=0 \Rightarrow$
$t_1=-\dfrac{3}{4}$ или $t_2=\dfrac{2}{3} \Rightarrow x^2-2x=-\dfrac{3}{4}$ или
$x^2-2x=-\dfrac{2}{3} \Rightarrow \\ x_1 = \dfrac12; x_2=\dfrac32; x_3=\dfrac{3-\sqrt3}2; x_4=\dfrac{3+\sqrt3}2$\\
Ответ: $\left\{ \dfrac12;\ \dfrac32;\ \dfrac{3\pm\sqrt3}{3} \right\}$

\label{Problem4}
\underline{Задача №4}
Решить уравнение $(x+2)(x+3)(x+8)(x+12)=4x^2$\newline
Решение: $\big((x+3)(x+8) \big) \big( (x+2)(x+12) \big)=4x^2 ; (x^2+11x+24)(x^2+14x+24)=4x^2 \Rightarrow 
\left( x+11+\dfrac{24}{x} \right) \left( x+14+\dfrac{24}{x} \right) = 4$, сделаем замену, 
$x+11+\dfrac{24}{x}=t$, $x+14+\dfrac{24}{x}=t+3 \Rightarrow t(t+3)=4 \Rightarrow$
$t_1=-4$ или $t_2=1 \Rightarrow x^2+11x+24=-4x$ или $x^2+11x+24=x \Rightarrow$
$x_{1,2}=\dfrac{-15+\sqrt{129}}{2}$; $x_{3,4}=-6; -4$\newline
Ответ: $\left\{ -6; -4; \dfrac{-15\pm\sqrt{129}}{2} \right\}$

\label{Problem5}
\underline{Задача №5}
Решить уравнение $(x^2-2x+2)^2+3x(x^2-2x+2)=10x^2$\newline
Решение: $\dfrac{(x^2-2x+2)^2}{x^2}+3\dfrac{(x^2-2x+2)}{x}=10$, 
$\dfrac{x^2-2x+2}{x}=t \Rightarrow t^2+3t=10 \Rightarrow t_1=2; t_2=-5 \Rightarrow$
$x^2-2x+2=2x$ или $x^2-2x+2=-5x \Rightarrow x_{1,2}=4\pm\sqrt2; x_3=-1; x_4=-2.$\\
Ответ: $\left\{ -1; -2; 4\pm\sqrt2 \right\}$

\label{Problem6}
\underline{Задача №6}
Решить уравнение $\dfrac{x^2}{2}+\dfrac{48}{x^2}=10\left(\dfrac{x}3-\dfrac{4}{x}\right)$\newline
Решение: $t=\dfrac{x}{3}-\dfrac{4}{x} \Rightarrow 3\left(t^2+\dfrac{8}{3}\right)=10t \Leftrightarrow$
$3t^2-10t+8=0 \Rightarrow t_1=2;t_2=\dfrac43 \Rightarrow x^2-12=6x$ или $x^2-12=4x \Rightarrow$
$x_{1,2}=3\pm\sqrt{21}; x_3=-2; x_4=6$\\
Ответ: $\{ -2; 6; 3\pm\sqrt{21} \}$
 
\label{Problem7}
\underline{Задача №7}
Решить уравнение $x^4-3x^3-8x^2+12x+16=0$\newline
Решение: $\dfrac{x^4}{x^2}-3\dfrac{x^3}{x^2}-8+\dfrac{12}{x}+\dfrac{16}{x^2}=0; $
$x^2-3x-8+\dfrac{12}{x}+\dfrac{16}{x^2}=0; x^2+\dfrac{4}{x^2}-3\left( x-\dfrac{4}{x} \right)-8=0,$
$t=x-\dfrac{4}{x} \Rightarrow t^2-3t=0 \Rightarrow t=0$ или $t=3 \Rightarrow$
$x^2-4=0$ или $x^2-4=3x \Rightarrow x_{1,2}=\pm2; x_3=4; x_4=-1$\\
Ответ: $\{ -1; 4; \pm2 \}$

\label{Problem8}
\underline{Задача №8} 
Решить уравнение $x^2+\dfrac{81x^2}{(x+9)^2}=40$\newline
Решение: $x^2+\dfrac{(9x)^2}{(x+9)^2}+\dfrac{2x\cdot9x}{x+9}-\dfrac{2x\cdot9x}{x+9}=40, $
$\left(x-\dfrac{9x}{x+9}\right)^2+18\dfrac{x^2}{x+9}=40,$
$\left( \dfrac{x^2}{x+9} \right)^2+18\dfrac{x^2}{x+9}=40, t=\dfrac{x^2}{x+9} \Rightarrow$
$t^2+18t-40=0 \Rightarrow t_1=2, t_2=-20 \Rightarrow x^2=2x+18$ или $x^2=-20x-180 \Rightarrow$
$x_{1,2}=1\pm\sqrt{19}$\\
Ответ: $\{ 1\pm\sqrt{19} \}$

\begin{center}
{\large Уравнения вида $a^4+b^4=(a+b)^4$}
\end{center}
 
Преобразуем выражение $a^4+b^4=(a+b)^4: a^4+b^4=(a^2+b^2+2ab)\cdot(a^2+b^2+2ab)=$
$a^4+b^4+\underbrace{2a^2b^2+4a^3b+4ab^3+4a^2b^2}_{\equiv 0}$\\
Очевидно, что тождество будет верным, когда $6a^2b^2+4a^3b+4ab^3=0 \Leftrightarrow\\ 2ab(3ab+2a^2+2b^2)=0 \Leftrightarrow$
$\left[\begin{gathered}
	a=0 \hfill \\
	b=0 \hfill \\
	3ab+2a^2+2b^2=0
\end{gathered}\right.$

\label{Problem9}
\underline{Задача №9} 
Решить уравнение $(x+1)^4+(x+3)^4=16(x+2)^4$\newline
Решение: $(x+1)^4+(x+3)^4=\big( 2(x+2) \big)^4=(2x+4)^4$. Пусть $a=x+1,b=x+3$, тогда
это уравнение эквивалентно:
$\left[\begin{gathered}
	x+1=0 \hfill \\
	x+3=0 \hfill \\
	2(x+1)^2+3(x+1)(x+3)+2(x+3)^2=0
\end{gathered}\right. \Rightarrow$
$\left[\begin{gathered}
	x_1=-1 \hfill \\
	x_2=-3 \hfill \\
	\varnothing
\end{gathered}\right.$\\

\begin{center}
{\large Сведение иррационального уравнения к системе уравнений}
\end{center}
 
\label{Problem10}
\underline{Задача №10} 
Решить уравнение $x^3+1=2\sqrt[3]{2x-1}$\newline
Решение: Пусть 
$\begin{dcases}
	x^3+1=2y \\
	2x-1=y^3
\end{dcases} ,$
$x^3-y^3+1=2y-2x+1 \Rightarrow$
$(x-y)(x^2+xy+y^2)=-2(x-y)$, следует, что если $x-y\ne0$, то $x^2+xy+y^2=-2 \Rightarrow \varnothing$, 
следовательно $x=y \Rightarrow x^3-2x+1=0 \Leftrightarrow (x-1)(x^+2x-1)=0 \Rightarrow$
$x_{1,2}=\dfrac{-1\pm\sqrt5}{2}; x_3=1$\\
Ответ: $\{ 1; \dfrac{-1\pm\sqrt5}{2} \}$

\label{Problem11}
\underline{Задача №11} 
Решить уравнение $\sqrt{x}+\sqrt[3]{x-3}=3$\newline
Решение: Пусть $u=\sqrt{x}$ и $v=\sqrt[3]{x-3} \Rightarrow$
$u+v=3$ и $v^3=u^2-3 \Rightarrow$
$\begin{dcases}
	u+v=3 \\
	v^3=u^2-3
\end{dcases} \Rightarrow$
$v^3=v^2-6v+6 \Leftrightarrow (v-1)(v^2+6)=0 \Rightarrow v=1 \Rightarrow \sqrt[3]{x-3}=1 $
$\Rightarrow x=4$\\
Ответ: $\{ 4 \}$

\label{Problem12}
\underline{Задача №12} 
Решить уравнение $\sqrt[4]{x+7}-\sqrt[4]{x-9}=2$\newline
Решение: Пусть $u=\sqrt[4]{x+7}$ и $v=\sqrt[4]{x-9}$, тогда
$\begin{dcases}
	u^4-v^4=16 \\
	u-v=2
\end{dcases} \Rightarrow$
$(2+v)^4-u^4=16 \Leftrightarrow $
$\left[\begin{gathered}
	v=0 \hfill \\
	v^2+3v+4=0
\end{gathered}\right. \Rightarrow$
$v=0 \Rightarrow u=2 \Rightarrow $
$\begin{dcases}
	\sqrt[4]{x+7}=2 \\
	\sqrt[4]{x-9}=0
\end{dcases} \Rightarrow$
$x=9$\\
Ответ: $\{ 9 \}$

\begin{center}
{\large Анализ ОДЗ}
\end{center}

\label{Problem13}
\underline{Задача №13} 
Решить уравнение $\sqrt{1-x^2}+\sqrt{x^2+x-2}+x+1=2x^3$\newline
Решение: Найдем ОДЗ 
$\begin{dcases}
	1-x^2>0 \\
	x^2+x-2 \ge 0
\end{dcases} \Rightarrow$
$\begin{dcases}
	x \in [-1;1] \\
	x \in (-\infty;-2] \cup [1;+\infty)
\end{dcases} \Rightarrow$
$x=1.$ Проверкой убеждаемся, что $x=1$ - корень уравнения\\
Ответ: $\{ 1 \}$

\label{Problem14}
\underline{Задача №14} 
Решить уравнение $\sqrt{x(x+1)}=\sqrt{x+3}-\sqrt{1+\dfrac{1}{x^2}}$\newline
Решение: ОДЗ уравнения - $x \in [-3;-1] \cup (0;+\infty)$, перепишем уравнение
$\sqrt{x(x+1)}+\sqrt{1+\dfrac{1}{x^2}}=\sqrt{x+3}$, возведем обе части уравнения
в квадрат, получим равносильное уравнение $2\sqrt{x(x+1)}\cdot\sqrt{1+\dfrac{1}{x^2}} \ge 0$,
получим систему: 
$\begin{dcases}
	x^2+\dfrac{1}{x^2}=2 \\
	\left[\begin{gathered}
		x(x+1)=0 \hfill \\
		1+\dfrac{1}{x^2}=0
	\end{gathered}\right.
\end{dcases} \Rightarrow x=-1$, проверкой убеждаемся, что это корень исходного уравнения.\\
Ответ: $\{ 1 \}$

\begin{center}
{\large Использование монотонности функций}
\end{center}

\label{Problem15}
\underline{Задача №15} 
Решить уравнение: $2\sqrt[3]{3x+2}+\sqrt[4]{x-1}=\sqrt{3-x}+4$\\
Решение: ОДЗ: $1\le x \le 3$, рассмотрим функцию $f(x)=2\sqrt[3]{3x+2}+\sqrt[4]{x-1}$,\\
$f'(x)=\dfrac{2}{\sqrt[3]{(3x+2)^2}}+\dfrac14\dfrac{1}{\sqrt[4]{(x-1)^3}}$
видим, что $f'(x)>0$ при $\forall x \in [1;3]$, следовательно $f(x) \nearrow(возрастающая)$.
$g(x)=\sqrt{3-x}+4$, $g'(x)=-\dfrac12\cdot \sqrt{1}{\sqrt{3-x}}$. 
$g'(x)<0 \forall x \in [1;3]$, следовательно, $g(x)\searrow$ (убывает).
Следовательно, уравнение имеет только один корень. Методом подбора определяем, что 
$x=2$\\
Ответ: $\{ 2 \}$

\begin{center}
{\large Использование числовых неравенств}
\end{center}

\label{Problem16}
\underline{Задача №16}
Решить уравнение: $(16x^{200}+1)(y^{200}+1)=16(xy)^{100}$\newline
Решение: Воспользуемся неравенством Коши: $\dfrac{16x^{200}+1}{2} \ge \sqrt{16x^{200}}$
и \\$\left( \dfrac{16x^{200}+1}{2} \right) \left( \dfrac{y^{200}+1}{2} \right) \le \sqrt{16x^{200}} \cdot \sqrt{y^{200}}$,
неравенство превращается в равенство при условии: 
$\begin{dcases}
	16x^{200}=1\\
	y^{200}=1
\end{dcases}\Rightarrow$
$\begin{dcases}
	x=\pm \dfrac{1}{\sqrt[50]{2}} \\
	y=\pm1
\end{dcases}$\\
Ответ: $\{ x=\pm\dfrac{1}{\sqrt[50]{2}}; y=\pm1 \}$

\label{Problem17}
\underline{Задача №17}
Решить уравнение: $\sqrt{1-x}+\sqrt{1+x}+\sqrt[4]{1-x^2}+\sqrt[4]{1+x^2}=4$\newline
Решение: Воспользуемся неравенством Бернулли $(1-x)^{1/2}+(1+x)^{1/2}+(1-x^2)^{1/4}+(1+x^2)^{1/4} \le 1-\dfrac12x+1+\dfrac12x+1-\dfrac14x^2+1+\dfrac14x^2$,
т.е. равенство достигается при $x=0$.\\
Ответ: $\{ 0 \}$

\begin{center}
{\large Некоторые дополнительные примеры}
\end{center}

\label{Problem18}
\underline{Задача №18}
Решить уравнение: $x=\underbrace{\sqrt{ 5+\sqrt{ 5+\sqrt{\ldots+\sqrt{5+x}} } }}_{n раз}$\newline
Решение: 1) ОДЗ: $x \le 0$\\
2) Пусть $f(x)=\sqrt{5+x}$, $f'(x)=\dfrac{1}{2\sqrt{5+x}}>0$, для $x>0$, т.е. $f(x) \nearrow$ (возрастает),
тогда решением этого уравнения будет $f(x)=x \Rightarrow x=\dfrac{1+\sqrt{21}}{2}$\\
Ответ: $\left\{ \dfrac{1+\sqrt{21}}{2} \right\}$

Уравнение вида $f(g(x))=f(h(x))$. Если $f(x)$ - строго возрастающая функция, то
уравнение равносильно уравнению $g(x)=h(x)$ на области допустимых значений. Если
$f(x)$ строго монотонная и четная, то уравнение равносильно
$	\left[\begin{gathered}
		g(x)=h(x) \hfill \\
		g(x)=-h(x)
	\end{gathered}\right.$
	
\label{Problem19}
\underline{Задача №19}
Решить уравнение: $(x-1)^4+4x-4=x^2+4\sqrt{x}$\newline
Решение: 1) ОДЗ: $x \le 0$, $(x-1)^4+4(x-1)=x^2+4\sqrt{x}$, пусть 
$f(x)=x^4+4x$, $g(x)=x-1$, $h(x)=\sqrt{x}$, тогда получим $f(g(x))=f(h(x))$,
$f'(x)=4x^3+4 > 0 \forall x>0$, т.е. $f(x) \nearrow$, следовательно уравнение
сводится к виду $x-1=\sqrt{x} \Rightarrow x=\dfrac{3+\sqrt5}{2}$\\
Ответ: $\left\{ \dfrac{3+\sqrt{5}}{2} \right\}$

\label{Problem20}
\underline{Задача №20}
Решить уравнение: $x^4-2x^2+2|x^2-1|+1=4x^2+4|x|$\newline
Решение: $(x^2-1)^2+2|x^2-1|=(2x)^2+2|2x|$, пусть $f(x)=x^2+2x$, $g(x)=x^2-1$ и 
$h(x)=2x$, тогда получим $f(g(x))=f(h(x))$, заметим также: $f(-x)=(-x)^2+2|-1|\cdot|x|=x^2+2x=f(x)$ - 
четная, при $x<0$ убывает; имеем:\\
$	\left[\begin{gathered}
	\begin{dcases}
		x \ge 0 \\
		\left[\begin{gathered}
			x^2-1=2x \hfill \\
			x^2-1=-2x
		\end{gathered}\right.
	\end{dcases}\\
	\begin{dcases}
		x \le 0 \\
		\left[\begin{gathered}
			x^2-1=2x \hfill \\
			x^2-1=-2x
		\end{gathered}\right.
	\end{dcases}	
	\end{gathered}\right. \Rightarrow$
$x=\pm1\pm\sqrt2$\\
Ответ: $\left\{ \pm1\pm\sqrt2 \right\}$

Решение уравнения вида $f(g(x))+f(h(x))=0$ сводится к решению уравнения вида
$f(g(x)) = f(-h(x))$, если $f(x)$ - нечетная.

\label{Problem21}
\underline{Задача №21}
Решить уравнение: $\sin\left( \dfrac{x}{x^2+1} \right) + \sin\left( \dfrac{1}{x^2+x+2} \right) = 0$\newline
Решение: Пусть $f(x)=\sin x$, $g(x)=\dfrac{x}{x^2+1}$, $h(x)=\dfrac{1}{x^2+x+2}$, тогда
получим $f(g(x))+f(h(x))=0$, т.к. $f(-x)=-\sin x=-\sin x=-f(x)$ - нечетная, тогда
$f(g(x)) = f(-h(x))$, $f'(x)=\cos x > 0$, для $-\dfrac{\pi}{2} < x < \dfrac{\pi}{2}$ - строго возрастает
$-\dfrac12 \le g(x) \le \dfrac12$, $0 \le h(x) \le \dfrac47$, следовательно уравнение сводится к решению
$\dfrac{x}{x^2+1}=-\dfrac{1}{x^2+x+2} \Rightarrow x^3+2x^2+2x+1=0 \Rightarrow x=-1$\\
Ответ: $\left\{ -1 \right\}$

\begin{center}
{\large Задачи для самостоятельного решения}
\end{center}

\begin{minipage}{0.5\linewidth}
	1) $\sqrt{2x+1}+\sqrt{x}=\sqrt{6x+1}$ \\
    2) $\sqrt{5x+7}-\sqrt{x+3}=\sqrt{3x+1}$ \\
    3) $\sqrt{x+1}-\sqrt{9-x}=\sqrt{2x-12}$ \\
    4) $\sqrt{x^3-x^2+4}+\sqrt{x^3-x^2+1}=3$ \\
    5) $\sqrt{x^2-6x+9}+\sqrt{x^2-4x+4}+\sqrt{4x^2-32x+64}=9$ \\
    6) $x^2-2x+4-3\sqrt{2x^2-4x+5}=-1$ \\
    7) $\sqrt{2x+\sqrt{6x^2+1}}=x+1$ \\
    8) $\sqrt{x}+\sqrt{x-\sqrt{1-x}}=1$ \\
    9) $\sqrt{3x-2}-\dfrac{x^2-2x+4}{\sqrt{3x-2}}=x-3$ \\
    10) $\sqrt{2x+3}+\sqrt{x-3}=3x-6+2\sqrt{2x^2-3x-9}$ \\
\end{minipage}
\begin{minipage}{0.5\linewidth}
	11) $\sqrt[3]{8+x}+\sqrt[3]{8-x}=1$ \\
	12) $\sqrt[3]{x-1}+\sqrt[3]{x-2}=\sqrt[3]{2x-3}$ \\
	13) $\sqrt{6-4x-x^2}=x+4$ \\
	14) $\sqrt{\dfrac{3-x}{2+x}}+3\sqrt{\dfrac{2+x}{2-x}}=4$\\
	15) $\sqrt[3]{x^2}-\sqrt[3]{x}-6=0$\\
	16) $\sqrt{3x^2-2x+15}+\sqrt{3x^2-2x+8}=7$ \\
	17) $\sqrt[4]{47-2x} + \sqrt[4]{35+2x}=4$ \\
	18) $(4x^2-9)\sqrt{x+1}=0$\\
	19) $(x-3)^2+3x-22=\sqrt{x^2-3x+7}$\\
	20) $\sqrt{x-2}+\sqrt{4-x}=x^2-6x+11$
\end{minipage}

\begin{center}
{\large Неравенства и олимпиадные задачи}
\end{center}

\label{Problem22}
\underline{Задача №22}
Решите неравенство: $\sqrt{6x-13}-\sqrt{3x^2-13x+13} \ge 3x^2-19x+26$. 
В ответе указать сумму всех удовлетворяющих неравенству целых значений $x$\\
Решение: $\sqrt{6x-13}=a$, $\sqrt{3x^2-13x+13}=b$, $a, b \ge 0 \Rightarrow$
$3x^2-19x+26=b^2-a^2$, $a-b \ge b^2-a^2 \Leftrightarrow (a-b)(a+b+1) \ge 0 \Rightarrow$
$a-b \ge 0 \Rightarrow a \ge b$, т.е. $\sqrt{6x-13} \ge \sqrt{3x^2-13x+13} \Leftrightarrow$
$\begin{dcases}
	6x-13 \ge 3x^2-13x+13 \\
	3x^2-13x+13 \ge 0
\end{dcases}, $ 
$x \in \left[ 2;\dfrac{13}{3} \right] ; x \in \left( \infty; \dfrac{13-\sqrt{13}}{6} \right] \cup \left[ \dfrac{13+\sqrt{13}}{6}; +\infty \right) \Rightarrow x=3;4$\\
Ответ: $7$

\label{Problem23}
\underline{Задача №23}
Найти все пары вещественных чисел $(x;y)$, удовлетворяющих системе:
$\begin{dcases}
	(\sqrt3-\sqrt2)^x=2^{y/3}+3^{y/3} \\
	\sqrt{-5x^2-8xy-3y^2}=-y-2x
\end{dcases}$\\
Решение: Разберемся сначала со вторым уравнением, возведем его в квадрат.
$-5x^2-8xy-3y^2=y^2+4xy+4x^2 \Leftrightarrow (2y+3x)^2=0 \Leftrightarrow $
$y=-\dfrac32 x$, пусть $-\dfrac{x}{2}=t$, тогда из первого уравнения:
$(\sqrt3-\sqrt2)^{-2t}=2^t+3^t$. Используя умножение на сопряженное выражение,
получаем: $(\sqrt3+\sqrt2)^{2t}=2^t+3^t$, пусть $b=(\sqrt3+\sqrt2)^2=5+2\sqrt6 > 3$, 
тогда $2^t+3^t=b^t \Leftrightarrow \left( \dfrac{2}{b} \right)^t + \left( \dfrac{3}{b} \right)^t=1$,
левая часть представляет сумму двух убывающих функций, т.к. основания меньше 1,
значит данное уравнение может иметь не более одного корня. Методом подбора
находим, что $t=\dfrac12 \Rightarrow x=-1, y=\dfrac32$.\\
Ответ: $\left(-1, \dfrac32\right)$

\label{Problem24}
\underline{Задача №24}
Решить неравенство $\dfrac{\sqrt{-x^2+x+6}}{|x^2-7x+6|-|x^2-x-2|} \ge 0$\newline
Решение: Т.к. неравенство $|f(x)|>|g(x)|$ равносильно каждому из неравенств
$f^2(x)>g^2(x)$, $(f(x)+g(x))(f(x)-g(x))>0$, то исходное неравенство равносильно
системе неравенств: 
$\begin{dcases}
	-x^2+x+9 \ge 0 \\
	(2x^2-8x+4)(-6x+8) > 0
\end{dcases}(1)$\\
Квадратный трехчлен $-x^2+x+6$ имеет корни $-2$ и $3$, корнями квадратного
трехчлена $x^2-4x+2$ являются числа $x_1=2-\sqrt2$ и $x_2=2+\sqrt2$, а системы $(1)$ равносильна
системе:
$\begin{dcases}
	(x+2)(x-3) \le 0 \hfill (2) \\
	(x-x_1)(x-\dfrac43)(x-x_2) < 0 \hfill (3)
\end{dcases}$, где $-2<x_1<\dfrac43<3<x_2$. Множество $E_1$ решений неравенства $(2)$ - отрезок
$-2 \le x \le 3$. Множество $E_2$ решений неравенства $(3)$, определяемое методом интервалов, является
объединением интервалов $x<x_1$ и $\dfrac43<x<x_2$, а множество решений системы $(2)$, $(3)$, - пересечение
множество $E_1$ и $E_2$.\\
Ответ: $[-2;2-\sqrt2) \cup \left(\dfrac43; 3\right]$

\begin{center}
{\large Задачи для самостоятельного решения}
\end{center}

\begin{minipage}{0.5\linewidth}
	21) $\dfrac{1}{2-\sqrt{x^2-3x}} \le \dfrac{1}{\sqrt{x^2-2x+4}}$ \\
	22) $\dfrac{1}{\sqrt{|x+1|-2}} \le \dfrac{1}{9+x}$
\end{minipage}
\begin{minipage}{0.5\linewidth}
	23) $\sqrt{3+4\cos^2x}=\dfrac{\sin x}{\sqrt3}+3\cos x$\\
	24) $\sqrt{x^2-5x+6} < 1 + \sqrt{x^2-x+1}$
\end{minipage}
\begin{center}
	25) $\sqrt{\dfrac{243+9x-2x^2}{2x+3}} > 9-|x|$
\end{center}
\end{document}
